\title{Lab6 STM32 Clock and Timer}
\author{0616069 張晉瑋, 0616014 楊政道}
\maketitle
\thispagestyle{fancy}
\section{Objectives}
\paragraph{}
\begin{enumerate}
    \item Understand the various clock source usage and modification of STM32
    \item Understand the principle of using STM32 timer
    \item Understand the principle and application of PWM for STM32
\end{enumerate}
\section{Experiment Procedure}
\subsection{Modify system initial clock}
\paragraph{}
We use SysTick interrupt to control the blinking of the LED on pin \texttt{PA5},
and poll for the button state and switch the system clock in the main loop.
\subsubsection{Initializations}
\paragraph{}
Our works for initializations included SysTick, GPIO, and max7219.
\paragraph{}
For SysTick, we configured registers \texttt{SYST\_CSR}(exception request enable and enable fields), \texttt{SYST\_RVR} so that the SysTick interrupt will be issued whenever the \texttt{SYST\_CVR} counts to 0.
The frequency for SysTick is the system clock divided by 8,
so the reload value \texttt{1000000} would give a blinking period of 1 second at 4MHz clock.
\begin{lstlisting}
SysTick->CTRL = 3;
SysTick->LOAD = 1000000;
\end{lstlisting}
\paragraph{}
For GPIO, we use PA5(on-board LED), PC13(user button) and PB3-5 connected with max7219 as debug console.
\begin{lstlisting}
void GPIO_init() {
	RCC->AHB2ENR |= RCC_AHB2ENR_GPIOAEN | RCC_AHB2ENR_GPIOBEN | RCC_AHB2ENR_GPIOCEN;

	GPIOA->MODER = (GPIOA->MODER & 0xFFFFF3FF) | 0x400;
	GPIOA->OSPEEDR = 0x800;

	GPIOB->MODER = (GPIOB->MODER & 0xFFFFF03F) | 0x540;
	GPIOB->OSPEEDR = 0xA80;

	GPIOC->MODER = (GPIOC->MODER & 0xF3FFFFFF);
	GPIOB->PUPDR = 0x04000000;
}
\end{lstlisting}
\subsubsection{Button Polling}
\paragraph{}
We rewrote the button polling code in C to improve the readibility.
The function \texttt{poll\_button()} returns 1 when the static integer counter \texttt{cnt} exceeds the given debounce threshold,
which means the button is pushed and the corresponding procedure should be executed.
\begin{lstlisting}
int poll_button() {
	static int cnt = 0;
	static int prev = 0;
	int status = GPIOC->IDR & (0x01 << 13);
	if(!status) {
		if(cnt > debounce[clk_state]) {
			if(prev == 0) {
				prev = 1;
				cnt = 0;
				return 1;
			}
			cnt = 0;
		}
		cnt++;
	} else {
		prev = 0;
		cnt = 0;
	}
	return 0;
}
int main() {
    /* ... */
	while(1) {
		if(poll_button()) {
            /* Procedure when button is pushed */
			clk_state++;
			if(clk_state == 5) clk_state = 0;
			switch_clk(PLLN[clk_state], PLLM[clk_state], PLLR[clk_state]);
		}
	}
    /* ... */
}
\end{lstlisting}
\subsubsection{Clock Switching}
\paragraph{}
In the requirement, we need to produce various clock frequency(6MHz, 10MHz, and 40MHz are relatively more special).
So we used PLL clock as our major system clock source in this lab.
\paragraph{}
The clock HSI, MSI, and HSE can be used as the source of PLL, 
we chose MSI at 4MHz since it is the original system clock, which doesn't need additional configurations.
The clock will be first divided by the prescalar \texttt{M},
multiplied by \texttt{N} in the feedback divider of the Phase Lock Loop,
and then divided by the prescalar \texttt{P, Q, R} for each output on their own.
The output \texttt{PLLCLK} depends on prescalar \texttt{R}, so given \texttt{n, m, r}, 
we are able to activate PLL clock we need and make it as the system clock, as the function \texttt{switch\_clk()} follows:
\begin{lstlisting}
void switch_clk(int n, int m, int r) {
	// switch to MSI clock first
	while(!(RCC->CR & RCC_CR_MSIRDY));
	RCC->CFGR = (RCC->CFGR & 0xFFFFFFFC) | 0x00;

	RCC->CR &= (0xFEFFFFFF);
	while(RCC->CR & RCC_CR_PLLRDY);
	RCC->PLLCFGR = (RCC->PLLCFGR & 0xF8FF808C) | (r << 25) | (n << 8) | (m << 4) | 0x01;
	RCC->CR |= 0x01000000;
	RCC->PLLCFGR |= 0x01000000;

	// switch to PLL clock
	while(!(RCC->CR & RCC_CR_PLLRDY));
	RCC->CFGR = (RCC->CFGR & 0xFFFFFFFC) | 0x03;
}
\end{lstlisting}
\paragraph{}
The above function turns the system clock into $4M \times n / m / r$Hz,
which does the following works:
\begin{enumerate}
    \item Switch to MSI clock(since the PLL clock should be turned off when modifying configurations)
    \begin{enumerate}
        \item Wait until MSI clock becomes stable.
        \item Swtich clcok into MSI clock.
    \end{enumerate}
    \item Set PLL parameters.
    \begin{enumerate}
        \item Disable PLL clock.
        \item Wait until PLLRDY bit is clear.
        \item Set \texttt{n, m} and \texttt{r} into corresponding fields in PLLCFGR.
        \item Enable PLL clock.
    \end{enumerate}
    \item Switch back to PLL clock.
    \begin{enumerate}
        \item Wait until PLL clock becomes stable.
        \item Swtich clcok into PLL clock.
    \end{enumerate}
\end{enumerate}
\paragraph{}
The parameters for each required frequency is given below:
\begin{itemize}
    \item  1MHz: n =  8, m = 4, r = 8, 4M $\times$  8 / 4 / 8 =  1MHz
    \item  6MHz: n = 24, m = 4, r = 4, 4M $\times$ 24 / 4 / 4 =  6MHz
    \item 10MHz: n = 40, m = 4, r = 4, 4M $\times$ 40 / 4 / 4 = 10MHz
    \item 16MHz: n = 64, m = 4, r = 4, 4M $\times$ 64 / 4 / 4 = 16MHz
    \item 40MHz: n = 40, m = 1, r = 4, 4M $\times$ 40 / 1 / 4 = 40MHz
\end{itemize}
\paragraph{}
Together with debounce counting thresholds:
\begin{itemize}
    \item  1MHz: 512
    \item  6MHz: 3072
    \item 10MHz: 5120
    \item 16MHz: 8192
    \item 40MHz: 20480
\end{itemize}

\subsection{Timer}
\subsubsection{Initializations}
\paragraph{}
We use the general purpose timer TIM2 in this lab. 
In order to count time from zero upwards to \texttt{TIME\_SEC} seconds in a single run(instead of repeating),
we configured the timer to run in one-pulse mode.
As we ready to start the timer, we issue an update event by setting the \texttt{UEV} bit in register \texttt{TIM2\_EGR},
then enable the timer by setting the \texttt{CEN} bit in register \texttt{TIM2\_CR1}
\paragraph{}
We also set the prescalar to be 3999 so that the counter is updated at each millisecond($4M/(3999+1) = 1000$),
so the reload value for the timer(\texttt{TIMx\_ARR}) can be simply set to \texttt{TIME\_SEC * 1000}.
Therefore the implementation of \texttt{timer\_init()} and \texttt{timer\_start()} is given as follows:
\begin{lstlisting}
void timer_init(TIM_TypeDef *timer) {
	RCC->APB1ENR1 |= 0x01;
	timer->PSC = 3999;
	timer->ARR = TIME_SEC * 1000;
	timer->CR1 = 0x09;
}

void timer_start(TIM_TypeDef *timer) {
	timer->EGR |= 0x01;
	timer->CR1 |= 0x01;
}
\end{lstlisting}
\subsubsection{Main Program and The Display of Time}
\paragraph{}
In the main program we just initialized the IO and timer, then start it.
After that we keep track of the \texttt{UIF} flag in the status register(\texttt{TIM2\_SR}) to check whether the counter overflows,
 we also display the counter value in max7219 in seconds with 2 decimal places in that loop.
\begin{lstlisting}
void display(int val) {
	max7219_send(0x01, val%10);
	val /= 10;
	max7219_send(0x02, val%10);
	val /= 10;
	max7219_send(0x03, 128 | val%10);
	val /= 10;
	for(int p=4; p<=8; p++) {
		if(val) {
			max7219_send(p, val%10);
		} else {
			max7219_send(p, 0x0F);
		}
		val/=10;
	}
}

int main() {
	GPIO_init();
	max7219_init();
	timer_init(TIM2);
	timer_start(TIM2);
	TIM2->SR &= (0xFFFFFFFE);
	while(1) {
		display(TIM2->CNT / 10);
		if(TIM2->SR & 1) {
			display(TIME_SEC * 100);
			break;
		}
	}
    while(1);

	return 0;
}
\end{lstlisting}

\subsection{Buzzer}
\subsubsection{Initializations}
\paragraph{}
In order to get PWM working and output the signal, we need additional configuration on both GPIO and the timer.
\paragraph{}
According to the STM32l476RG datasheet,
the output of the timer channel \texttt{TIM2\_CH1} is set on the alternative function \texttt{AF1} of pin \texttt{PA0}.
Thus for the GPIO, we need to set the alternative function mode in the mode register(\texttt{GPIOA\_MODER}) and the required alternative function in the register \texttt{GPIOA\_AFRL/GPIOA\_AFRH}.
\paragraph{}
For the timer, we don't use the prescalar and one-pulse mode,
and the reload values corresponding to each frequency is stored in the array \texttt{pitch\_arr} and will be set into \texttt{TIM2\_CCR1} at pitch updates.
In addition, register \texttt{TIM2\_CCMR2} and \texttt{TIM2\_CCER1} is configured so that the output compare 1 is enabled.
So the function \texttt{timer\_init()} became as below:
\begin{lstlisting}
void timer_init(TIM_TypeDef *timer) {
	RCC->APB1ENR1 |= 0x01;
	timer->PSC = 0;
	timer->CCMR1 = 0x00000060;
	timer->CCER |= 0x01;
}
\end{lstlisting}

\subsubsection{Updating the Pitch}
\paragraph{}
We scan the keypad just as lab 5.
When the number is lower than or equal to 8, we set the pitch by the function \texttt{timer\_set\_pitch()}.
This function simply check if the reload value changes, and updates it if so.
The \texttt{UEV} bit in register \texttt{EGR} is set at the end in order to clear the counter value so that \texttt{TIMx\_CNT < TIMx\_ARR} always holds:
\begin{lstlisting}
void timer_set_pitch(TIM_TypeDef *timer, int arr) {
	static int current_arr = 0;
	if(arr == current_arr) {
		return ;
	}
	current_arr = arr;
	timer->ARR = arr;
	timer->CCR1 = arr/2;
	timer->EGR |= 0x01;
}

int main() {
    /* Initialization */
    /* ... */
    /* While scanning */
    /* ... */
        int r = 9-i + 3*j;
        if(i == 5)r = 10 + j;
        if(j == 3)r = 8 + i;
        if(r == 15) r = 0;
        if(r == 16) r = 15;
        if(r <= 8)timer_set_pitch(TIM2, pitch_arr[r], cycle);
    /* ... */
    /* When scanning is finished and no buttons are pressed */
    if(!pressed) {
        timer_set_pitch(TIM2, 0);
    }
}
\end{lstlisting}

\subsection{Modify LED Brightness}
\paragraph{}
This is basically similar with the lab 6-3, with two buttons that changes the duty cycle.
we keep another variable for the current duty cycle,
and change it using the function \texttt{timer\_set\_pitch()} with additional parameter \texttt{cycle} as follows:
\begin{lstlisting}
void timer_set_pitch(TIM_TypeDef *timer, int arr, int cycle) {
	static int current_arr = 0;
	if(arr == current_arr) {
		return ;
	}
	if(arr != -1) {
		current_arr = arr;
		timer->ARR = arr;
	}
	timer->CCR1 = current_arr*cycle/100;
	timer->EGR |= 0x01;
}
\end{lstlisting}
The determination of the two keypad button is modified like this:
\begin{lstlisting}
/* ... */
/* When scanning */
    if((GPIOB->IDR >> j) & 0x40) {
        int r = 9-i + 3*j;
        if(i == 5)r = 10 + j;
        if(j == 3)r = 8 + i;
        if(r == 15) r = 0;
        if(r == 16) r = 15;
        if(r <= 8)timer_set_pitch(TIM2, pitch_arr[r], cycle);
        else if(r == 10 && !hold[0]) {
            hold[0] = 1;
            cycle += 10;
            if(cycle > 90)cycle = 90;
            timer_set_pitch(TIM2, -1, cycle);
        } else if(r == 11 && !hold[1]) {
            hold[1] = 1;
            cycle -= 10;
            if(cycle < 10)cycle = 10;
            timer_set_pitch(TIM2, -1, cycle);
        }
        pressed = 1;
    }
/* ... */
/* When scanning is finished and no buttons are pressed */
if(!pressed) {
    hold[0] = hold[1] = 0;
    timer_set_pitch(TIM2, 0, cycle);
}
\end{lstlisting}
\section{Feedback}
\subsection{PLL Clock Configuration}
\paragraph{}
In lab 6-1, we were determining proper values for \texttt{N, M, R} so that the PLL clock would run at the required frequency.
Knowing that the value of \texttt{M} ranges from 1 to 8, we set these values into \texttt{PLLCFGR} as same as their bit representation.
It results in wierd behavior while setting PLL as system clock, 
and took us several hour to figure out that \texttt{PLLM} field only have 3 bits.
\subsection{MAX7219 as Debug Console}
\paragraph{}
Outputting intermediate results is a common trick in debugging. 
With \texttt{max7219\_send()} function, we are able to do such work when the cases aren't that complicated.
However, such approach doesn't work when the clock frequency exceeds 10MHz, which is the maximum working frequency of max7219.
This told us that timing restrictions should be taken into account when the MCU cooperates with other devices, 
instead of using highest available frquencies directly.
