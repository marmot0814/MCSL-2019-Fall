\title{Lab6 STM32 Clock and Timer}
\author{0616069 張晉瑋, 0616014 楊政道}
\maketitle
\thispagestyle{fancy}
\section{What 實驗要做什麼}
\paragraph{}
\section{How 實驗過程}
\paragraph{}
\section{Feedback}
\subsection{PLL Clock Configuration}
\paragraph{}
In lab 6-1, we were determining proper values for \texttt{N, M, R} so that the PLL clock would run at the required frequency.
Knowing that the value of \texttt{M} ranges from 1 to 8, we set these values into \texttt{PLLCFGR} as same as their bit representation.
It results in wierd behavior while setting PLL as system clock, 
and took us several hour to figure out that \texttt{PLLM} field only have 3 bits.
\subsection{MAX7219 as Debug Console}
\paragraph{}
Outputting intermediate results is a common trick in debugging. 
With \texttt{max7219\_send()} function, we are able to do such work when the cases aren't that complicated.
However, such approach doesn't work when the clock frequency exceeds 10MHz, which is the maximum working frequency of max7219.
This told us that timing restrictions should be taken into account when the MCU cooperates with other devices, 
instead of using highest available frquencies directly.
