\title{Lab8 USART and ADC}
\author{0616069 張晉瑋, 0616014 楊政道}
\maketitle
\thispagestyle{fancy}
\section{Objectives}
\paragraph{}
The objectives of this lab are to learn the mechanism of USART communication, including transmit and recieve,
and the ADC unit of STM32, which provides the conversion from analog voltage to digital integral values.
\section{Experiment Procedure}
\subsection{Hello World!}
\paragraph{}
There are several USART communication channels.
To enable them, we need to set up the APB they belong to and their own control registers.
In order to make it connected with external devices, the alternative function should also be set.
Then the USART transmitter sends the data in \texttt{USART\_TDR} and clears \texttt{TXE} bit at each single bit transmission.
\paragraph{}
In this lab, we use USART1 on GPIO pin \texttt{PA9, PA10},
so the registers of GPIO and \texttt{APB2ENR, USART\_CR1, USART\_BRR} are configured as follows:
\begin{lstlisting}
void GPIO_init() {
	RCC->AHB2ENR |= RCC_AHB2ENR_GPIOAEN | RCC_AHB2ENR_GPIOCEN;
	GPIOA->MODER = (GPIOA->MODER & 0xFFC3FFFF) | 0x280000;
	GPIOA->AFR[1] = (GPIOA->AFR[1] & 0xFFFFF00F) | 0x770;
}
void ConfigUSART() {
	RCC->APB2ENR |= RCC_APB2ENR_USART1EN;
	USART1->BRR = 0x1A0;
	USART1->CR1 |= USART_CR1_TE;
	USART1->CR1 |= USART_CR1_UE;
}
\end{lstlisting}
\paragraph{}
The main loop simply polls for the button and the \texttt{TXE} bit:
\begin{lstlisting}
int main() {
	GPIO_init();
	ConfigUSART();
	while(1) {
		if(poll_button()) {
			for(int i=0; str[i]; i++) {`
				while(!(USART1->ISR & USART_ISR_TXE));
				USART1->TDR = str[i];
			}
		}
	}
	return 0;
}
\end{lstlisting}
\subsection{ADC}
\paragraph{}
In this section, we will get an analog signal, voltage between photoresistance and ground,  convert the analog signal into digital signal by ADC and show it on our personal computer through USART protocal. 
\paragraph{}
GPIO about USART is same to the lab8-1. We use PC0 to get the analog signal and set its mode into analog mode.
\begin{lstlisting}
void GPIO_init() {
    RCC->AHB2ENR |= RCC_AHB2ENR_GPIOAEN;
    GPIOA->MODER = (GPIOA->MODER & 0xFFC3FFFF) | 0x280000;
    GPIOA->AFR[1] = (GPIOA->AFR[1] & 0xFFFFF00F) | 0x770;

    RCC->AHB2ENR |= RCC_AHB2ENR_GPIOCEN;
    GPIOC->MODER = (GPIOC->MODER & ~(0x3 << (2 * 13))) | (0x0 << (2 * 13));   // PC13 input mode
    GPIOC->MODER = (GPIOC->MODER & 0xFFFFFFFC) | 0x3;
    GPIOC->PUPDR = 0xAA;
    GPIOC->ASCR |= 1;
}
\end{lstlisting}
\paragraph{}
Then, we need to initial ADC device.
\begin{lstlisting}
void ADC1_init() {
	RCC->AHB2ENR |= RCC_AHB2ENR_ADCEN;              // enable ADC
    ADC1->CFGR &= ~ADC_CFGR_RES;                    // resolution 12-bit
    ADC1->CFGR &= ~ADC_CFGR_CONT;                   // single conversion mode
    ADC1->CFGR &= ~ADC_CFGR_ALIGN;                  // right alignment

    ADC123_COMMON->CCR &= ~ADC_CCR_DUAL;            // independent mode
    ADC123_COMMON->CCR &= ~ADC_CCR_CKMODE;          // HCLK / 1
    ADC123_COMMON->CCR |= 1 << ADC_CCR_CKMODE_Pos;
    ADC123_COMMON->CCR &= ~ADC_CCR_PRESC;           // prescaler: div 1
    ADC123_COMMON->CCR &= ~ADC_CCR_MDMA;            // disable DMA
    ADC123_COMMON->CCR &= ~ADC_CCR_DELAY;
    ADC123_COMMON->CCR |= 4 << ADC_CCR_DELAY_Pos;

    ADC1->SQR1 &= ~(ADC_SQR1_SQ1);                  // channel: 1, rank: 1
    ADC1->SQR1 |= 1 << ADC_SQR1_SQ1_Pos;

    ADC1->SMPR1 &= ~ADC_SMPR1_SMP1;                 // ADC sample pre 6.5 clock cycle
    ADC1->SMPR1 |= 2 << ADC_SMPR1_SMP1_Pos;

    ADC1->CR &= ~ADC_CR_DEEPPWD;                    // disable deeppwd
    ADC1->CR |= ADC_CR_ADVREGEN;                    // enable ADC voltage regulator
    for (int i = 0 ; i < 1000 ; i++);               // wait for voltage regulator
    ADC1->IER |= ADC_IER_EOCIE;                     // enable end of conversion interrupt
    NVIC_EnableIRQ(ADC1_2_IRQn);
    ADC1->CR |= ADC_CR_ADEN;                        // enable ADC
    while (!(ADC1->ISR & ADC_ISR_ADRDY));           // wait for ADC startup
}
\end{lstlisting}
\paragraph{}
First, we need to enable ADC on AHB2ENR and set some parameters. After we setup all the parameters, we will disable deeppwd, enable ADC voltage regulator, EOC interrupt and ADC.
\paragraph{}
The whole control flow is that we start adc conversion after getting a SysTick interrupt. After the conversion complete, ADC will throw an EOC interrupt and get the voltage data from ADC->DR. When the user poll the button, we will send the voltage value to the target computer and display on its monitor.
\begin{lstlisting}
int voltage;

#pragma thumb
void SysTick_Handler() {
    ADC1->CR |= ADC_CR_ADSTART;                     // start adc conversion
}

#pragma thumb
void ADC1_2_IRQHandler() {
	while (!(ADC1->ISR & ADC_ISR_EOC));             // wait for conversion complete
	voltage = (int) ADC1->DR;
}
void print(char *s) {
    for(int i=0; s[i]; i++) {
        while(!(USART1->ISR & USART_ISR_TXE));
        USART1->TDR = s[i];
    }
}

void printInt(int tar) {
    static char buf[100]; buf[0] = '\0'; int ptr = 0;
    while (tar)
        buf[ptr++] = (tar % 10) + '0', tar /= 10;
    if (ptr == 0)
        buf[ptr++] = '0';
    buf[ptr] = '\0';
    int L = 0, R = ptr - 1;
    while (L < R) {
        char tmp = buf[L]; buf[L] = buf[R]; buf[R] = tmp;
        L++; R--;
    }
    print(buf);
}

int main() {
    GPIO_init();
    ADC1_init();
    USART_init();
    SysTick_init();
    while (1) {
        if (poll_button())
            print("\r                    \rvoltage: "), printInt(voltage);
    }
}
\end{lstlisting}

\subsection{Simple Shell}
\paragraph{}
Compared with lab 8-1, we need to configure USART reciever additionally.
We also need to configure ADC as lab 8-2
\section{Feedback}
\paragraph{}
