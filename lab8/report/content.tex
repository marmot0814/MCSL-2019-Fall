\title{Lab8 USART and ADC}
\author{0616069 張晉瑋, 0616014 楊政道}
\maketitle
\thispagestyle{fancy}
\section{Objectives}
\paragraph{}
The objectives of this lab are to learn the mechanism of USART communication, including transmit and recieve,
and the ADC unit of STM32, which provides the conversion from analog voltage to digital integral values.
\section{Experiment Procedure}
\subsection{Hello World!}
\paragraph{}
There are several USART communication channels.
To enable them, we need to set up the APB they belong to and their own control registers.
In order to make it connected with external devices, the alternative function should also be set.
Then the USART transmitter sends the data in \texttt{USART\_TDR} and clears \texttt{TXE} bit at each single bit transmission.
\paragraph{}
In this lab, we use USART1 on GPIO pin \texttt{PA9, PA10},
so the registers of GPIO and \texttt{APB2ENR, USART\_CR1, USART\_BRR} are configured as follows:
\begin{lstlisting}
void GPIO_init() {
	RCC->AHB2ENR |= RCC_AHB2ENR_GPIOAEN | RCC_AHB2ENR_GPIOCEN;
	GPIOA->MODER = (GPIOA->MODER & 0xFFC3FFFF) | 0x280000;
	GPIOA->AFR[1] = (GPIOA->AFR[1] & 0xFFFFF00F) | 0x770;
}
void ConfigUSART() {
	RCC->APB2ENR |= RCC_APB2ENR_USART1EN;
	USART1->BRR = 0x1A0;
	USART1->CR1 |= USART_CR1_TE;
	USART1->CR1 |= USART_CR1_UE;
}
\end{lstlisting}
\paragraph{}
The main loop simply polls for the button and the \texttt{TXE} bit:
\begin{lstlisting}
int main() {
	GPIO_init();
	ConfigUSART();
	while(1) {
		if(poll_button()) {
			for(int i=0; str[i]; i++) {
				while(!(USART1->ISR & USART_ISR_TXE));
				USART1->TDR = str[i];
			}
		}
	}
	return 0;
}
\end{lstlisting}

\subsection{Simple Shell}
\paragraph{}
Compared with lab 8-1, we need to configure USART reciever additionally.
We also need to configure ADC as lab 8-2
\section{Feedback}
\paragraph{}
