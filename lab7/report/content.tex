\title{Lab7 Interrupt and Exception}
\author{0616069 張晉瑋, 0616014 楊政道}
\maketitle
\thispagestyle{fancy}
\section{Objectives}
\paragraph{}
In this lab, we are going to learn the procedure to activate different types of exceptions in STM32 system.
This includes NVIC and SysTick in Cortex M4 processor, and the EXTI in STM32, which sends signals into NVIC.
We are going to use these featuers to implement programs that interact with user based on interrupts instead of polling values in previous labs.
\section{Experiment Procedure}
\subsection{SysTick Timer Interrupt Setting}
\paragraph{}
The whole program flow is simple. We enable the HSI clock and switch to it after it was ready.
Then set reload value(\texttt{SysTick->LOAD}) and control register that enables counter and its interrupt(\texttt{SysTick->CTRL}).
The handler simply toggles the state of the output:
\begin{lstlisting}
void SysTick_Handler() {
	GPIOA->ODR = GPIOA->ODR ^ (1<<5);
}
\end{lstlisting}
We use the default clock source, which is the AHB clock with the \texttt{/8} prescalar for SysTick.
In such manner the reload value of \texttt{6000000} gives the 3 second period(6M/(16M/8)).

\section{Feedback}
\subsection{Clearing the Pending Bits}
\paragraph{}
The pending bit in EXTI pending registers seem to stay high after the handler returns, and should be cleared manually.
We are still not sure about if it is the actual behavior, and also don't know the reason for such design.
\subsection{Other Approaches to Obtain The Pressed Key}
\paragraph{}
In lab 7-2, we were looking for possible ways to obtain what key is pressed when the interrupts are triggered, which includes:
\begin{itemize}
\item Rescan the whole keypad in the EXTI handlers.
\item Read IDR in the EXTI handlers.
\item Use EXTI line 0 to 4, which have their own seperated handlers in the vector table. This is the approach we used.
\item Read the active bit or pending bit in the handler.
\end{itemize}
Although the third method is easier to implement,
we thought the last may be most appropriate among these methods,
since the bits really indicates the actual state of the exceptions, 
instead of the signal source, which may not keep generating the signal that are still easy to detect.
It is better than the third option since the slots for exception in vector table seems to be fairly expensive,
also the responding procedures are almost the same in each handler.
