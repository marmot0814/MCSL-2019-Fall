\title{Lab2 ARM Assembly 2}
\author{0616069 張晉瑋, 0616014 楊政道}
\maketitle
\thispagestyle{fancy}
\section{Objectives}
\paragraph{}
As the specification, implement postfix arithmic and greatest common divisor program, and use stack manupulation in ARM Assembly.
\section{Experiment Procedure}
\subsection{Postfix Arithmic}
\paragraph{}
We are not pushing \texttt{LR} into stack since we are not going back to \texttt{\_start} in this program, and the procedure \texttt{atoi} doesn't call any other procedure.
\paragraph{}
There are two stack can be used in Cortex M4, which are the main stack and the process stack. The address of the two stack are stored in \texttt{MSP} and \texttt{PSP} and one can use \texttt{MSR} instruction to switch between them.
\paragraph{}
To set up the stack to be on the \texttt{.data} section we defined, we set the register \texttt{PSP} to be the \textbf{end} of the symbol \texttt{user\_stack} since the stack grows downward in address.
\paragraph{}
In the main program, we keep the current character and the next character in registers \texttt{R2, R3}, and make the read address in \texttt{R0} point at the next two byte. When reading each character we move \texttt{R3} to \texttt{R2} and then load the byte at \texttt{[R0]} to \texttt{R3}. In such manner we are able to determine whether the current character is a sign or an operator. If the current character is a digit or a sign, then we branch into \texttt{atoi} routine, otherwise we pop the top two item in the stack into register \texttt{R4, R5}, perform the corresponding operation and push the result back to the stack. We skip the spaces after we processed current token. The branching scheme is as following code:
\begin{lstlisting}
    loop_1:
    teq r2, #0x00
    beq loop_1_end
    teq r3, #' '
    beq op_start
    teq r3, #0x00
    bne read_number
op_start:
    teq r2, #'+'
    beq op_add
    teq r2, #'-'
    beq op_sub

op_add:
    // doing addition and push result back
op_sub:
    // doing subtraction and push result back
op_end:
    ldrb r2, [r0], #1
    ldrb r3, [r0], #1
    b loop_1

read_number:
    bl atoi
    push {r1}
    b loop_1
\end{lstlisting}

\paragraph{}
In the \texttt{atoi} routine, since the address pointing the string is 2 chracters ahead, we use the offset \texttt{[R0, \#-2]} rather than continuing the look ahead scheme. We store the sign and magnitude of the value and append the digit when we read it. Finally we multiply the magnitude by the sign and then return. The address and the register value for the look ahead scheme is restored afterwards. The code for \texttt{atoi} without initialization is given below, where \texttt{R3, R5} store the character \texttt{'0'} and number $10$ :
\begin{lstlisting}
    loop_2:
    teq r2, #' '
    beq loop_2_end
    teq r2, #0x00
    beq loop_2_end
    teq r2, #'-'
    bne append_digit
    ldrb r2, [r0, #-2]
    adds r0, #0x01
    mov r4, #-1
append_digit:
    muls r1, r1, r5
    subs r2, r2, r3
    adds r1, r1, r2
    ldrb r2, [r0, #-2]
    adds r0, #0x01
    b loop_2
loop_2_end:
    muls r1, r1, r4
    subs r0, #2
    ldrb r2, [r0], #1
    ldrb r3, [r0], #1
    bx lr
\end{lstlisting}
\subsection{Greatest Common Divisor with Stein's Algorithm}
\paragraph{}
In this section, we need to implement Stein's Algorithm to calculate the greatest common divisor between given two numbers by recursive method. To implement the recursive function, we need to manipulate the system stack a.k.a call stack to maintain the return address and function arguments of the previous function call. Therefore, the main structure of the recursive function will be like this below
\begin{lstlisting}
func:
    push {<some argument>, lr}
    // if this is a terminated state, goto terminated state label
    // content of the recurrence relation
    bl func
    pop  {<some argument>, pc}
teriminatedState:
    // content of the terminated state
    pop  {<some argument>, pc}
\end{lstlisting}
\paragraph{}
The next task we need to complete is the content of the terminated state and the recurrence relation.
\paragraph{}
The content of the terminated state is a trivial case.
\begin{equation}
  gcd(m, n)=\begin{cases}
    m, & \text{if n is zero}.\\
    n, & \text{if m is zero}.\\
    \text{<recurrence relation>}, & \text{otherwise}.
  \end{cases}
\end{equation}
\paragraph{}
which is one of the two numbers is zero and the other number is the greatest common divisor between the two numbers. We can use some branch and compare instructions to deal with this case. The code is listed below as an attachment.
\begin{lstlisting}
    cmp R0, #0                  // if (r0 == 0)
    beq gcdR1Ret                // goto return r1 case
    cmp R1, #0                  // if (r1 == 0)
    beq gcdR0Ret                // goto return r0 case
    b cntn                      // goto recurence relation case
gcdR1Ret:
    mul R2, R2, R1              // multiply answer by r1
    b gcdRet                    // return answer
gcdR0Ret:
    mul R2, R2, R0              // multiply answer by r0
    b gcdRet                    // return answer
gcdRet:
    pop  {R0, R1, pc}           // return function
\end{lstlisting}
\paragraph{}
The content of the recurrece relation can be divided into four cases.
\begin{equation}
  gcd(m, n)=\begin{cases}
    m, & \text{if n is zero}.\\
    n, & \text{if m is zero}.\\
    2 * gcd(m / 2, n / 2), & \text{if m is even and n is even}.\\
    gcd(m / 2, n), & \text{if m is even and n is odd}.\\
    gcd(m, n / 2), & \text{if m is odd and n is even}.\\
    gcd(m, abs(m - n)), & \text{if m is odd and n is odd}.
  \end{cases}
\end{equation}
\paragraph{}
We need to construct a if...else structure and do the corresponding action according to each cases. The code is listed below as an attachment.
\begin{lstlisting}
    and R3, R0, #1              // r3 = r0 & 1
    and R4, R1, #1              // r4 = r1 & 1
    orr R5, R3, R4              // r5 = r3 | r4
    cmp R5, #0                  // if (r5 == 0)
    bne else1                   //
    lsr R0, R0, #1              // r0 /= 2
    lsr R1, R1, #1              // r1 /= 2
    lsl R2, R2, #1              // r2 *= 2
    b endif
else1:
    cmp R3, #0                  // if (r3 == 0)
    bne else2
    lsr R0, R0, #1              // r0 /= 2
    b endif
else2:
    cmp R4, #0                  // if (r4 == 0)
    bne else3
    lsr R1, R1, #1              // r1 /= 2
    b endif
else3:
    sub R3, R0, R1              // r3 = r0 - r1
    cmp R3, #0                  // if (r3 < 0)
    bgt skipAbs
    mov R4, #0                  //
    sub R3, R4, R3              //   r3 = 0 - r3
skipAbs:
    mov R4, R1                  // r4 = r1
    cmp R0, R1                  // if (r0 < r1)
    bgt isR1
    mov R4, R0                  //   r4 = r0
isR1:
    mov R0, R3                  // r0 = r3
    mov R1, R4                  // r1 = r4
    b endif
endif:
    bl gcd
    pop  {R0, R1, pc}
\end{lstlisting}
\paragraph{}
Using the way above, we can get the greatest common divisor between two numbers and store the value into the target memory address. The maximum size of the system stack usage can be done by the following method. First, store the initial stack pointer into a register. Then, we can maintain the minimun value of the stack pointer after the push instruction. Finally, we can substract the minimun value of the stack pointer from the initial stack poiner to get the maximum size of the system stack usage. It can be done by just adding some codes after the push instruction and the code is listed below as an attachment.
\begin{lstlisting}
    push {R0, R1, lr}
    cmp R7, sp                  // compare stack pointer
    blt skipUpdateSp
    mov R7, sp
skipUpdateSp:
    <the content of the recurrsive function>
\end{lstlisting}
\section{Feedback}
\paragraph{}
