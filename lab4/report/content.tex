\title{Lab4 7-Segment LED}
\author{0616069 張晉瑋, 0616014 楊政道}
\maketitle
\thispagestyle{fancy}
\section{Objectives}
\paragraph{}
With the help of GPIO, we are able to craft variable signals for external devices.
In this lab, the MAX7219 LED display driver runs the role.
We're going to control GPIO pins to communicate with MAX7219, and display the messages assigned in the spec.

\section{Experiment Procedure}
\subsection{Sending the Packet}
\paragraph{}
As lab note stated, the signals sent to MAX7219 is synchronized by the \texttt{CLK} pin, 
and the data on \texttt{DIN} pin is stored into the shift registers on it.
When pin \texttt{CS} rises to high level, MAX7219 latches the content in the shift register and store it into corresponding registers according to the address field in the packet.
\paragraph{}
Since MAX7219 only recognizes rising edge of the signal of pin \texttt{CLK},
and the system clock(4MHz) is slower than the maximum allowed frequency of MAX7219,
we don't need to perform any delay to make each clock cycle equal.
With given address in register \texttt{r0} and data in \texttt{r1} as parameters,
we constructed the function \texttt{MAX7219Send} as follows:
\begin{listlst}

\end{listlst}
\subsection{4-1 MAX7219 in non-decode mode}
\paragraph{}
\section{Feedback 實驗心得}
\paragraph{}
