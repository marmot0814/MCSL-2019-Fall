\title{Lab3 STM32 GPIO System}
\author{0616069 張晉瑋, 0616014 楊政道}
\maketitle
\thispagestyle{fancy}
\section{Objectives}
\paragraph{}
In this lab, we're going to use general purpose IO ports and make programs that interacts with user,
which includes LED outputs, push button and DIP switch inputs.

\section{Experiment Procedures}
\subsection{The Initialization for GPIO}
\paragraph{}
As a peripherals, the GPIO system involves controls to the bridge it belongs to, which is AHB,
thus we need to enable its clock to drive the GPIO system.
For the GPIO system itself, we need to set up its mode(input/output), type, speed, and pull up/down resistors in our case.
All of the registers involving these attributes are mapped to their corresponding memory addresses, 
thus the value can be easily set by load/store instructions.
\paragraph{}
We give the following code to show the initialization of an output pin,
the procedure for an input pin is similar with difference in IO mode, the input/output register to read/write,
and pull up/down mode in some cases.
\begin{lstlisting}
GPIO_init:
    push    {lr}
    bl      set_RCC_AHB2ENR
    bl      set_GPIO_MODER
    bl      set_GPIO_OSPEEDR
    pop     {pc}

set_RCC_AHB2ENR:
    mov     r1,     #0x2                                //  Set PB on
    ldr     r0,     =RCC_AHB2ENR                        //  Load RCC address
    str     r1,     [r0]                                //  Store 0x2 into RCC
    bx      lr  

set_GPIO_MODER:
    ldr     r0,     =GPIOB_MODER                        //  Load MODER address
    ldr     r1,     [r0]                                //  Load MODER current value
    and     r1,     #0xFFFFC03F                         //  Clear target address
    orr     r1,     #0x00001540                         //  Write into target address
    ldr     r0,     =GPIOB_MODER                        //  Load MODER address
    str     r1,     [r0]                                //  Store back to MODER address
    bx      lr  

set_GPIO_OSPEEDR:
    mov     r1,     #0x800                              //  Set ouput speed
    ldr     r0,     =GPIOB_OSPEEDR                      //  Load OSPEEDR address
    strh    r1,     [r0]                                //  Store back to OSPEEDR address
    bx      lr  
\end{lstlisting}

\subsection{Pattern}
\paragraph{}
We connect \texttt{3.3V} pin to the anodes of the four LEDs, and connect a resistor in series for each LED. 
The cathode of them are connected to \texttt{PB3}~\texttt{PB6}.
In such manner, the output will be in active-low logic.

\paragraph{}
For the configuration of GPIO registers,
we set four consecutive pins as output to make setting values in output register by shifting possible.
The \texttt{leds} variable represents the shift amount.
In a single round, it goes from \texttt{0} to \texttt{4}, then go back to \texttt{0}.
This is done by checking the value of \texttt{leds} and adding the difference \texttt{mvr} at each step.
Then the value of the ouput register will be \texttt{0x0C << leds}.
The code below is how we mutate \texttt{leds}, where \texttt{mvr} is always $1$ or $-1$:

\begin{lstlisting}
    ldr     r0,     =mvr                                //  Load mover address
    ldr     r2,     [r0]                                //  Load mover value
    ldr     r0,     =leds                               //  Load leds offset address
    ldr     r1,     [r0]                                //  Load leds offset value
    add     r1,     r1,     r2                          //  Update offset value
    ldr     r0,     =leds                               //  Load leds offset address
    str     r1,     [r0]                                //  Store back to leds offset address
                                                        //  Update move value
    cmp     r1,     #0                                  //  If leds offset value equal to 0
    beq     ReverseMVR                                  //      Reverse mover
    cmp     r1,     #4                                  //  If leds offset value equal to 4
    beq     ReverseMVR                                  //      Reverse mover
    bx      lr
\end{lstlisting}

\paragraph{}
The remain work became timing. We've known that the \texttt{SYSCLK} is 4MHz at the reset state.
Therefore we may keep a counter, change the pattern when the counter counts to \texttt{0}.
Since the countdown and check procedure consumes fixed CPU cycles, we may calculate the counter value from the number of such cycles.

\subsection{Push Button}
\paragraph{}
The point for this lab is debounce. We keep a counter and add it down as we read the high level on the pin.
When its value come above the threshold set by us, we consider it to be pressed and set a variable to indicate that it is changed.
\subsection{Password}
In this lab, we have 4 more inputs than the previous one, but the four inputs don't need debouncing.
So this is basically the combination of lab 3-1, 3-2 with some program state transition.

\section{Feedback}
\subsection{The Attempt for Timer}
\paragraph{}
In the beginning, we intended to use \texttt{SysTick} handler to change the pattern of the LEDs,
so that we would not need to implement busy waiting on our own.
We kept failing until we almost finished the whole lab,
and found out that \texttt{.thumb\_func} should be placed on handler symbol declarations,
or it will assert a UsageFault.
We didn't use this solution since it still need some modification.
Hope it made us more familiar with timers in the future.

\subsection{Maintainability}
\paragraph{}
The control flow of the program became more complex, since it needs to wait for input or output rather than doing all the work on its own.
Consequently, we need to keep the code maintainable and seperate independent works into subroutines.
Therefore we use stack to store more state instead of leaving everything messy in the registers, 
and the calling procedure is becoming more like a calling convention.
